%%%%%%%%%%%%%%%%%%%%%%%%%%%%%%%%%%%%%%%%%%%%%%%%%%%%%
%% 					Acronymes						%
%%%%%%%%%%%%%%%%%%%%%%%%%%%%%%%%%%%%%%%%%%%%%%%%%%%%%

%\newacronym{asb}{ASB}{bande de cisaillement adiabatique --ou \emph{Adiabatic Shear Band}--}

\newglossaryentry{upmc}{
	type=\acronymtype,
	name={UPMC},
	description={Université Pierre et Marie Curie},
	first={Université Pierre et Marie Curie (UPMC)},
	}
	
\newglossaryentry{cmap}{
	type=\acronymtype,
	name={CMAP},
	description={Centre de mathématiques appliquées de l'école Polytechnique},
	first={Centre de mathématiques appliquées de l'école Polytechnique (CMAP)},
	}
	
\newglossaryentry{iraa}{
	type=\acronymtype,
	name={IRAA},
	description={Institut de recherche sur l'architecture antique},
	first={Institut de recherche sur l'architecture antique (IRAA)},
	}
	
\newglossaryentry{cnrs}{
	type=\acronymtype,
	name={CNRS},
	description={Centre national de recherche scientifique},
	}
	
\newglossaryentry{iscd}{
	type=\acronymtype,
	name={ISCD},
	description={Institut des sciences du calcul et des données},
	first={Institut des sciences du calcul et des données (ISCD)},
	}
	
\newglossaryentry{rir}{
	type=\acronymtype,
	name={RIR},
	description={Réponse impulsionnelle d'une salle, ou  \emph{Room Impulse Response}},
	first={réponse impulsionnelle d'une salle, ou  \emph{Room Impulse Response} (RIR)},
	}

\newglossaryentry{fir}{
	type=\acronymtype,
	name={FIR},
	description={Filtre à réponse impulsionnelle finie, ou  \emph{Finite Impulse Response filter}},
	first={Filtre à réponse impulsionnelle finie, ou  \emph{Finite Impulse Response filter} (FIR)},
	}	
%\newacronym{iscd}{ISCD}{Institut des sciences du calcul et des données}


%%%%%%%%%%%%%%%%%%%%%%%%%%%%%%%%%%%%%%%%%%%%%%%%%%%%%
%% 			Définitions (glossaires standard)		%
%%%%%%%%%%%%%%%%%%%%%%%%%%%%%%%%%%%%%%%%%%%%%%%%%%%%%

%% BLENDER 

\newglossaryentry{array}{
	name={modifier "Array"},
	description={Permet de répéter n fois un objet en disposant les copies dans le repère absolu, le repère relatif à l'objet source ou bien par rapport à un objet tiers. Il est par exemple utilisé pour créer des escaliers en répétant n fois la première marche. Pour répéter l'objet selon une courbe, on peut lier le modifier à un objet vide (empty) qui aura subit une rotation. Cela est utilisé par exemple pour repéter les colonne du Porticus in Summa Cavea.},
	first={modifier "Array" qui permet de répéter n fois un objet en disposant les copies dans le repère absolu, le repère relatif à l'objet source ou bien par rapport à un objet tiers}
	}
	

\newglossaryentry{boolean}{
	name={modifier "Boolean"},
	description={Permet des opérations d'addition, de soustraction ou d'intersection entre objets. Il est utilisé par exemple pour faire des trous dans les murs pour les portes.},
	first={modifier "Boolean" qui permet des opérations d'addition, de soustraction ou d'intersection entre objets}
	}
	
%% ARCHEO

\newglossaryentry{aditus}{
	name={\scshape{Aditus}},
	text={aditus},
	description={Portes conduisant de l'exterieur à l'\gls{orchestra}},
	}
	
\newglossaryentry{architrave}{
	name={Architrave},
	description={Support horizontal posé au-dessus de colonnes ou d'un fronton},
	}

\newglossaryentry{basilique}{
	name={Basilique},
	description={Large pièce de forme quasi-carré qui flanque le mur de scène et les \gls{parascaenium}},
	}
		
\newglossaryentry{cavea}{
	name={Cavea},
	description={Désigne l'ensemble des rangées concentriques composant les gradins},
	}
	
\newglossaryentry{cuneus}{
	name={\scshape{Cuneus}},
	text={cuneus},
	description={Groupe de gradins representant une portion de la \gls{cavea}},
	}	
	
\newglossaryentry{chapiteau}{
	name={Chapiteau},
	description={Evasement au sommet d'une colonne},
	}	
\newglossaryentry{hyposcaenium}{
	name={\scshape{Hyposcaenium}},
	text={hyposcaenium},
	description={Fosse situé sous la scène comportant notamment le mécanisme du rideau de scène},
	}
	
\newglossaryentry{maenianum}{
	name={Maenianum},
	description={Portions de la cavea séparées par un \gls{podium} et rassemblant un ensemble de gradins},
	}
\newglossaryentry{odeon}{
	name={\scshape{Odéon}},
	text={odéon},
	description={Petit théâtre couvert dédié exclusivement aux spectacles musicaux},
	}	
	
\newglossaryentry{orchestra}{
	name={Orchestra},
	description={Espace semi-circulaire (chez les romains) ou circulaire (chez les Grecs) se situant entre la scène et le premier gradin},
	}	

\newglossaryentry{parascaenium}{
	name={Parascaenium},
	description={Espace intermédiaire entre la scène et les basilliques comportant des escaliers pour atteindre les niveaux supérieurs},
	}
		
\newglossaryentry{pilastre}{
	name={Pilastre},
	description={Faux pilier intégré au mur en ornement},
	}
\newglossaryentry{podium}{
	name={Podium},
	description={Massif de maçonnerie élevé au-dessus du sol et servant de soubassement},
	}

\newglossaryentry{postscaenium}{
	name={Postscaenium},
	description={Mur séparant la scène de l'extérieur comportant des salles pouvant servir de coulisses},
	}

\newglossaryentry{precinction}{
	name={Precinction},
	description={Palier situé au-dessus de chaque \gls{maenianum} et sur lequel s’ouvre les vomitorium},
	}

\newglossaryentry{porticus isc}{
	name={Porticus in summa cavea},
	description={Arcade bordée de colonnes située au dessus du troisième \gls{maenianum}},
	}	

\newglossaryentry{porticus ps}{
	name={Porticus post scaenam},
	description={Arcade bordée de colonnes située à l'extérieur du théâtre et adossée au mur de scène},
	}	

\newglossaryentry{velum}{
	name={\scshape{Velum}},
	text={velum},
	description={Grande pièce de tissu généralement en lin tirée au dessus de la \gls{cavea} pour protéger les spectateurs du soleil},
	}

\newglossaryentry{pulpitum}{
	name={Pulpitum},
	description={Petit mur bordant la scène côté orchestre},
	}

\newglossaryentry{vomitorium}{
	name={Vomitorium},
	description={Issues permettant aux spectateurs d'accéder aux gradins},
	}
	
%%%%%%%%%%%%%%%%%%%%%%%%%%%%%%%%%%%%%%%%%%%%%%%%%%%%%
%%					Symboles						%	
%%%%%%%%%%%%%%%%%%%%%%%%%%%%%%%%%%%%%%%%%%%%%%%%%%%%%
\newglossaryentry{alpha}{
  type=notation,
  name={\ensuremath{\alpha}},
  description={Angle de d'attaque de la molette},
  sort={alpha}%
}

\newglossaryentry{gamma}{
  type=notation,
  name={\ensuremath{\gamma}},
  description={Angle de dépouille de la molette},
  sort={gamma}
}

