\chapter*{Conclusion générale}
	\addcontentsline{toc}{chapter}{Conclusion générale}
	
	
Tout au long de ce projet nous avons décortiqué l'architecture du théâtre d'Orange. Ce monument, bien que partiellement restauré aujourd'hui a conservé de nombreux indices nous ayant permis d'en proposer une restitution virtuelle. Ainsi, nous disposons d'une maquette représentant le théâtre d'Orange dans son dernier état d'utilisation antique. Celle-ci, disponible sous \textit{Blender}, mais exportable en de nombreux formats, est destinée à être corrigée et enrichie lors des prochaines études archéologiques. Elle est en tout cas largement référencée dans ce document et ouvre de nombreuses possibilités de recherche. Certaines d'entre elles ont déjà été entamées dans le cadre du projet SONAT. Ainsi, s'appuyant sur la structure modélisée, les équipes de l'\gls{iscd} ont pu proposer des restitutions réalistes du déploiement du \gls{velum} ou du rideau de scène par exemple. Cette maquette virtuelle est donc désormais amenée à vivre au fil des recherches et a été conçue dans ce but. En effet, la plupart des objets virtuels la composant sont restés dans un état très simple avec des \glspl{modifier} permettant d'y appliquer des détails de manière non permanente. Cela rend sa manipulation simple et non destructive.

Ce projet a également donné naissance à un nouvel outil de prédiction acoustique utilisant des méthodes de calcul géométrique. En approximation hautes fréquences, l'algorithme génère la réponse impulsionnelle de salle en propageant des rayons dans toutes les directions et en créant des sources-images. La réverbération de la salle peut alors être étudiée dans le temps et dans l'espace. Cet outil générique s'interface directement sur le logiciel de \gls{cao} \textit{Blender} mais peut également fonctionner simplement à partir d'un fichier de maillage. Ce lien avec \textit{Blender} permet de modifier très rapidement les paramètres ou la géométrie d'une salle avec tout le panel d'options que fournit \textit{Blender}. L'algorithme est par ailleurs optimisé grâce à une structure hiérarchique utilisant des \glspl{octree}, ce qui le rend très rapide. Un cas complexe de plusieurs centaines de milliers d'éléments tel que le théâtre d'Orange peut donc être traité en moins de dix minutes. À titre de comparaison, le principal logiciel équivalent du marché, \textit{Odeon}, met plus de vingt minutes rien que pour charger le maillage puis traite le calcul total dans un temps équivalent ou supérieur selon les configurations. Cela rend notre outil très avantageux du point de vue de la simplicité et de la vitesse d'exécution. Il est par ailleurs multiplateforme, ce qui le rend utilisable sur 100\% des ordinateurs et ainsi, accessible au plus grand nombre. Les utilisateurs peuvent donc étudier, visualiser et écouter la réverbération d'une salle quelconque afin de faire naître de nouvelles perspectives archéologiques.

Le développement de ces deux briques a permis l'analyse acoustique du théâtre d'Orange dans une version restituée. L'utilisation des technologies virtuelles donne à ce procédé toute sa puissance puisque les manipulations sont très rapides et les tests sont multipliables facilement. Nous avons donc pu analyser le théâtre d'Orange dans différentes configurations : avec ou sans toit au dessus de la scène, avec différents taux de remplissage de public, pour des sources ou des spectateurs en différents points, etc. Nous avons compris que la clarté de la voix n'est pas excellente contrairement à ce qui se dit traditionnellement. Celle-ci est dégradée par les éléments qui ont aujourd'hui disparu dans la plupart des théâtres antiques et se dégrade également lorsqu'on s'éloigne de la scène ou lorsqu'on se trouve sur les ailes de la \gls{cavea}. Enfin, certains résultats lancent de nouveaux débats archéologiques, comme par exemple : quel était le rôle de la niche située au dessus de la porte royale ? Où se situaient les meilleures positions pour déclamer du texte ou pour jouer de la musique ?

Ce projet ouvre donc de nouvelles perspectives sur l'analyse archéologique du théâtre d'Orange. Tout d'abord, la maquette virtuelle pourra être complétée et des tests physiques pourront y être menés (portance du toit, prise en compte et modélisation du vent, etc). Les outils d'analyse pourront également être optimisés afin de faciliter le travail des utilisateurs non-acousticiens. Apporter des affichages plus intuitifs \cite[]{immersive} ou simplifier l'interface homme-machine seraient donc de bonnes idées d'amélioration. Par ailleurs, la vitesse de calcul pourrait encore être augmentée en utilisant un parallélisme multi-processeurs. Le logiciel doit être pris en main par des beta-testeurs afin d'en dégager les éventuels bugs résiduels et de faire naitre de nouvelles idées d'amélioration. En outre, différentes options pourraient être ajoutées, comme par exemple les effets de vibro-acoustique ou bien la diffraction et la transmissivité de certains matériaux. Les tests acoustiques dans le théâtre d'Orange pourront être poursuivis au fil des idées archéologiques avec par exemple des sources directionnelles, des matériaux plus proches de la réalité ou de nouvelles positions d'écoute. \\

Dans ce projet, la plus grande difficulté était sans doute de faire cohabiter plusieurs disciplines aux antipodes les unes des autres et pourtant si liées. Ainsi, à une problématique archéologique, nous avons apporté une solution mathématique en utilisant les lois de l'acoustique. Cela fut permis par l'utilisation judicieuse d'outils numériques et le développement d'algorithmes performants. Il s'agit donc d'un bel exemple de ce que la pluridisciplinarité peut apporter à la science. Ce projet laisse désormais une grande liberté et un grand champ d'application possible quant à la poursuite de l'étude virtuelle, visuelle et auditive du théâtre antique d'Orange.

% Biblio
 \bibliographystyle{francaissc}
 \bibliography{Part3/Biblio}
%\addcontentsline{toc}{chapter}{Références}