\chapter*{Conclusion}
	\addcontentsline{toc}{chapter}{Conclusion}
	
	
Tout au long de ce projet nous avons décortiqué l'architecture du théâtre d'Orange. Ce monument, bien que partiellement restauré aujourd'hui a conservé de nombreux indices nous ayant permis d'en proposer une restitution virtuelle. Ainsi, nous disposons d'une maquette virtuelle représentant le théâtre d'Orange dans son dernier état d'utilisation antique. Celle-ci, disponible sous \textit{Blender}, mais exportable en de nombreux formats est destinée à être corrigée et enrichies par les prochaines études archéologiques. Elle est en tout cas largement référencée dans ce document et ouvre de nombreuses possibilités de recherche. Certaines d'entre elles ont déjà commencées dans le cadre du projet SONAT. Ainsi, s'appuyant sur la structure modélisées, les équipes de \gls{iscd} ont pu proposer des restitutions réaliste du déploiement du \gls{velum} ou du rideau de scène par exemple. Cette maquette virtuelle est donc désormais amenée à vivre au fils des recheches et a été conçue dans ce but. En effet, la plupart des objets virtuels la composant sont resté dans un état très simple avec des \glspl{modificateur} permettant d'y appliquer des détails de manière non permanente.

Ce projet a également donné naissance a un nouvel outil de calcul acoustique utilisant des méthodes de calcul géométriques. Par approximation hautes fréquences l'algorithme génère la réponse impulsionnelle de salle en propageant des rayons dans toutes les directions de l'espace et en créant des sources-images. La réverbération de la salle peut alors être étudié dans le temps et dans l'espace. Cet outil générique s'interface directement sur le logiciel de \gls{CAO} \textit{Blender} mais pourrait fonctionner simplement à partir d'un fichier de maillage. Ce lien avec \textit{Blender} permet de pouvoir modifier très rapidement les paramètres ou la géométrie de la salle à tester avec tout le panel d'options que fourni \textit{Blender}. L'algorithme est par ailleurs optimisé grâce à des \gls{octree} ce qui le rend très rapide. Un cas complexe de plusieurs centaines de milliers d'éléments tel que le théâtre d'Orange peut donc être traité en moins de dix minutes. À titre de comparaison le principal logiciel équivalent du marché, Odéon met plus de dix rien que pour charger le maillage et a un temps de calcul total supérieur. Cela rend notre outil très avantageux du point de vue de la vitesse d'execution. Il est par ailleurs multiplateforme, ce qui le rend utilisable sur 100\% des ordinateurs. 