\chapter*{Remerciements}

J'écris ces lignes à l'issue d'une grande aventure qui aura duré trois années et qui auraient été impossible sans l'aide, le soutient et la patience de certaines personnes. Je tenais donc à leur adresser mes remerciements. Tout d'abord je souhaite m'adresser à Emmanuelle Rosso et Pascal Frey, pour avoir pensé à ce projet ambitieux et m'avoir fait confiance pour le mener à bien. Qui aurait cru qu'un ingénieur en opto-électronique aurait pu prendre à bras le corps un projet mêlant mathématiques appliquées, archéologie, acoustique et bien d'autres. Dans la continuité, je remercie également chaleureusement François Alouges de l'école Polytechnique pour son suivi réconfortant et pour avoir su apporter les mots justes tout au long du projet. 

J'aurais été on ne peut plus perdu sans l'aide précieuse et l'oeil d'expert-archéologue de Titien Bartette mon collègue de l'ISCD qui m'a largement guidé aux travers des vestiges du théâtre d'Orange. Je dois également un grand merci à François Salmon du CMAP pour ses conseils et explications en développement logiciel et traitement du signal de même qu'à Loïc Norgeot de l'ISCD pour la partie \textit{Blender} et visualisation. Ils ont tous deux su m'apporter leur expérience au moment opportun. Je n'oublie pas Sébastien Le Gall qui m'a émerveillé en sublimant graphiquement la maquette du théâtre d'Orange que j'avais réalisé. 

Je voudrais également remercier Alain Badie, Jean-Charles Moretti, Dominique Tardy et le reste de l'équipe de l'IRAA pour leurs retours d'experts et pour m'avoir fourni leurs relevés et documents indispensables à la réalisation du modèle numérique du théâtre d'Orange. Merci à Jean-Dominique Polack du LAM pour ses cours d'acoustique de salle et pour avoir partager avec nous ses connaissances sur le théâtre d'Orange ainsi qu'une partie de son matériel.

Merci à Delphine Aubry, Marie Duboué, Alexia Maximin, Noël Dimarcq, Tristan Briant, Guillaume Reuiller pour m'avoir accueillis et épaulés lors de passionnantes missions doctorales de conseil ou de médiation scientifique.

Je tiens également à apporter mes remerciement à mes collègues de l'ISCD et du CMAP pour leur bonne humeur .

Bien sûr, j'ai gardé le meilleur pour la fin, tel l'apothéose d'un spectacle. Je dois un énorme merci à Matthieu Aussal du CMAP car rien de tout cela n'aurai été possible sans lui. Je sais qu'il appréciera la métaphore en disant qu'il fut pour moi un véritable "bâton de berger" durant ces trois années.

Je profite par ailleurs de ce préambule pour remercier ma famille, parents et amis. Je remercie tout particulièrement Sophie pour son soutient quotidien et ses encouragements continus. Elle eu le courage de démarrer la gestation de notre fils en même temps que je commençais celle de ce document. Mon fils arriva environ à la page 100 pour notre plus grand bonheur. Merci Martin pour tes réveils nocturnes me forçant à poursuivre ma rédaction.

Finalement je vous remercie vous, chers lecteurs de tenter l'aventure de lire ce document. J'imagine qu'il est peu probable que vous soyez mathématicien spécialiste en archéologie, architecture, modélisation graphique, acoustique et algorithmique, c'est pourquoi j'ai essayé d'être le plus clair possible dans chacune de mes parties pour permettre à chacun d'accéder sans encombre aux sujets où il sera néophyte. Dans tous les cas, je vous souhaite une bonne lecture.


\newpage
\chapter*{Résumé}
Ce projet pluridisciplinaire initié par Sorbonne Université vise à apporter des solutions mathématiques et informatiques à des problèmes archéologiques. Il s'articule autour du théâtre antique d'Orange. Ce monument grandiose vieux de plus de 2000 est l'un des mieux conservé au monde. Il est principalement composé d'une quarantaine de gradins disposés en "U" et fermés par un large mur de scène. Une étude documentaire poussée a permis de restituer ce monument sur le logiciel \textit{Blender}. La maquette virtuelle du théâtre est composée de formes brutes, auxquelles sont affectées des éléments de détail non permanents, ce qui la rend très facilement modifiable. Précise à l'échelle du centimètre, elle concatène un grand nombre de relevés architecturaux et sert de base aux futurs travaux archéologiques.

 En plus de pouvoir visualiser virtuellement le monument, ce projet vise à étudier son acoustique. Ainsi, un outil de calcul spécialement conçu pour ce cas d'application complexe a été développé en C++. Celui-ci s'interface directement à \textit{Blender}, rendant les manipulations ergonomiques. L'algorithme est développé à l'aide de méthodes géométriques en se plaçant dans l'approximation hautes fréquences. À partir d'une source sonore, il s'agit de propager des faisceaux portant une certaine quantité d'énergie dans toutes les directions et de calculer leurs réflexions sur les parois du bâtiment. Pour un récepteur positionné dans l'espace, on peut alors connaitre la réponses impulsionnelle et l'emplacement des sources-images correspondantes. La méthode permet donc de connaitre la réverbération d'une salle sur huit bandes de fréquence (62 à 15000Hz) en prenant en compte les coefficients d'absorption des matériaux et du milieu de propagation. L'algorithme est optimisé par une approche de \textit{Divide and Conquer} utilisant des \textit{octrees}. Cela permet de réduire la complexité quadratique de l'interaction rayons/éléments à quasi-linéaire ce qui améliore considérablement le temps de calcul. L'algorithme est validé par comparaison avec des cas tests théoriques.
 
C'est avec ce simulateur acoustique qu'est menée l'étude du théâtre d'Orange. Grâce à différentes données de sortie (numériques, visuelles ou auditives) il est possible de qualifier l'impact acoustique des certaines configurations du théâtre : présence de décor ou de toit, position des spectateurs ou des sources sonores. Ce travail permet d'ouvrir sur de nouvelles perspectives archéologiques.



%\newpage
\chapter*{Abstract}
This multidisciplinary project initiated by Sorbonne University aims to provide mathematical and computing solutions to archaeological problems. It deals with the ancient theatre of Orange. This magnificent monument, more than 2000 years old, is one of the best preserved in the world. It is mainly composed of about forty steps arranged in a "U" shape and closed by a large stage wall. An extensive literature review allowed to restore this monument thanks to the software \textit{Blender}. The virtual model of the theatre is composed of coarse objets, detailed by non-permanent modifiers, which makes it really easy to update. Accurate to the centimetre scale, it concatenates a large number of architectural schemes and serves as a basis for future archaeological work.

In addition to visualization part, this project also aims to study the acoustics of the monument. Thus, a numerical simulation tool specially designed for this significant size numerical problems has been developed in C++. It has been designed to fit to Blender, making the manipulations ergonomic. The algorithm uses geometric methods thanks to the high frequency approximation. From a sound source, beams carrying a certain amount of energy are propagated in all directionsand reflected on the building walls. For a punctual receiver positioned in the 3D space, we can obtain the room impulse response and the location of the corresponding image-sources. The method provide the reverberation curve of a room for eight octave bands by considering the materials proprieties and the propagation medium. The algorithm is optimized by using a Divide and Conquer approach with a recursive octree structure. This allow to reduce the quadratic complexity of the ray/element interactions to near-linear and significantly improves computation time. The algorithm is validated by comparison with theoretical test cases.
 
The study of the Orange theatre is carried out with this acoustic simulator. Thanks to different output data (digital, visual or aural) it is possible to qualify the acoustic impact of certain theatre configurations: presence of ornament or roof, position of spectators or sound sources. This work opens up new archaeological perspectives.


%This article presents new numerical simulation tools, both for Matlab and Blender CAD software. Available in open-source under GPL 3.0 license, it uses a ray tracing / image-sources hybrid method to calculate room acoustics for large meshes. Performances are optimized to solve significant size numerical problems (typically more than 100,000 surface elements and about a million of rays). For this purpose, a \textit{Divide and Conquer} approach with a recursive octree structure has been implemented to reduce the quadratic complexity of the ray/element interactions to near-linear. Thus, execution times are less sensitive to mesh density, which allows complex geometry simulations. After ray propagation, a hybrid method leads to image-sources format, which can be visually analyzed to localize sound map. Finally, impulse responses are constructed from the image-sources and FIR filters are proposed natively over 8 octave bands, taking into account material absorption properties and propagation medium. This algorithm is validated by various comparison with theoretical test cases. Furthermore, an exemple on a complex case with the ancient theater of Orange is presented. 