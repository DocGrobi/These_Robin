\chapter*{Préface de l'auteur}

Cher lecteur,


avant de parcourir ce manuscrit, fruit de trois années de travail, je tenais à faire les présentations. Je m'appelle Robin Gueguen et je viens de vous soumettre ce document intitulé "Virtualisation architecturale visuelle et auditive du théâtre antique d'Orange". Ce travail interdisciplinaire est pour le moins atypique, et à projet atypique, profil atypique. Après deux années de classes préparatoires et trois années d'école d'ingénieur j'obtiens mon diplôme avec une spécialisation en optronique. Je participe ensuite aux développement de LIDAR (Radar-Laser) utilisés pour la mesure atmosphérique au sein de la société Leosphere. Rattaché au service de recherche et développement et particulièrement à la section optoélectronique j'acquière pendant quatre année l'expérience du monde industriel. En 2015 j'apprend l'existence de la collaboration entre l'Institut du Calcul et de la Simulation (ancien nom de l'ISCD) et de la Sorbonne autour de la virtualisation du théâtre d'Orange. Passionné aussi bien par les arts graphiques que les arts du spectacle, le projet me séduit tout de suite et je quitte mon emploi pour démarrer ces trois années de thèse. Je découvre alors le monde de l'archéologie et architecture mais aussi celui de la modélisation 3D et du développement logiciel. Je me forme aux principes de l'acoustique de salle et jongle avec les différentes attentes que chaque spécialité. J'imagine qu'il est peu probable que ce document ne trouve un lecteur mathématicien spécialiste en archéologie, architecture, modélisation graphique, acoustique, algorithmie. C'est pourquoi je vais tenter d'être le plus clair possible dans chacune de mes parties pour permettre à chacun d'accéder sans encombre aux sujets où il sera néophytes. L'enjeu de ce projet est surtout d'entrevoir une chaine complète de développement avec à la base : ce monument antique, et suite à un ensemble complexe de manipulations mathématiques en obtenir de nouvelles informations archéologiques jusqu'alors inaccessibles. 

\newpage
\chapter*{Remerciements}

Tout d'abord je souhaite remercier Emmanuelle Rosso et Pascal Frey mes deux directeurs de thèse pour avoir pensé à ce projet ambitieux et m'avoir fait confiance pour le mener à bien. Je remercie également chaleureusement François Alouges de l'école Polytechnique pour son suivi réconfortant et pour avoir su apporter les mots juste tout au long du projet. Pour son aide précieuse et son oeil expert d'archéologue je remercie mon collègue de l'ISCD Titien Bartette qui m'a largement guidée au travers des vestiges du théâtre d'Orange. Un grand merci à François Salmon du CMAP pour ses conseils et explication en développement logiciel et traitement du signal de même qu'à Loïc Norgeot de l'ISCD pour la partie blender et visualisation. Ils ont tous deux su m'apporter leur expérience au moment opportun. 

Je voudrait également remercier Alain Badie et l'équipe de l'IRAA pour m'avoir fourni leurs relevés et documents indispensables à la réalisation du modèle numérique du théâtre d'Orange. Merci à Jean-Dominique Polack du LAM pour ses cours d'acoustique de salle et pour avoir partager avec nous ses connaissances du le théâtre d'Orange et une partie de son matériel.

Merci à Delphine Aubry, Marie Duboué, Alexia Maximin, Noël Dimarcq, Tristan Briant, Guillaume Reuiller pour m'avoir accueilli et épaulé lors de passionnante mission doctorales de conseil ou de médiation scientifique.

Merci également à les collégues de l'ISCD pour leur bonne humeur : Lydie, Chiara, Sylvia, Sebastien, Agnès, Paule, Delphine, Christelle, Auréliane, Jérémie.

Bien sur rien de tout cela n'aurai été possible sans Matthieu Aussal du CMAP qui fut pour moi (et je crois qu'on ne peut pas mieux le décrire) un véritable "bâton de berger" durant ces trois années.

Merci à la merveilleuse Doudou pour ses encouragements continus sa grande beauté son sens de l'humour tonitruant !

