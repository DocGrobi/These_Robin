\chapter*{Remerciements}

J'écris ces lignes à l'issue d'une grande aventure qui aura duré trois années et qui auraient été impossible sans l'aide, le soutient et la patience de certaines personnes. Je tenais donc à leur adresser mes remerciement. Tout d'abord je souhaite m'adresser à Emmanuelle Rosso et Pascal Frey, mes deux directeurs de thèse pour avoir pensé à ce projet ambitieux et m'avoir fait confiance pour le mener à bien. Qui aurait cru qu'un ingénieur en opto-électronique aurait pu prendre à bras le corps un projet mêlant mathématiques appliquées, archéologie, acoustique et bien d'autres. Dans la continuité, je remercie également chaleureusement François Alouges de l'école Polytechnique pour son suivi réconfortant et pour avoir su apporter les mots justes tout au long du projet. 

J'aurais était on ne peut plus perdu sans l'aide précieuse et l'oeil d'expert archéologue de Titien Bartette mon collègue de l'ISCD qui m'a largement guidée au travers des vestiges du théâtre d'Orange. Je dois également un grand merci à François Salmon du CMAP pour ses conseils et explication en développement logiciel et traitement du signal de même qu'à Loïc Norgeot de l'ISCD pour la partie Blender et visualisation. Ils ont tous deux su m'apporter leur expérience au moment opportun. Je n'oublie pas Sébastien Le Gall qui m'a émerveillé en sublimant graphiquement la maquette du théâtre d'Orange que j'avais réalisé. 

Je voudrait également remercier Alain Badie, Jean-Charles Moretti, Dominique Tardy et le reste de l'équipe de l'IRAA pour leurs retours d'experts et pour m'avoir fourni leurs relevés et documents indispensables à la réalisation du modèle numérique du théâtre d'Orange. Merci à Jean-Dominique Polack du LAM pour ses cours d'acoustique de salle et pour avoir partager avec nous ses connaissances du le théâtre d'Orange ainsi qu'une partie de son matériel.

Merci à Delphine Aubry, Marie Duboué, Alexia Maximin, Noël Dimarcq, Tristan Briant, Guillaume Reuiller pour m'avoir accueilli et épaulés lors de passionnantes missions doctorales de conseil ou de médiation scientifique.

Je tiens également à apporter mes remerciement à mes collègues de l'ISCD et du CMAP pour leur bonne humeur .

Bien sûr, j'ai gardé le meilleur pour la fin tel l'apothéose d'un spectacle. Je dois un énorme merci à Matthieu Aussal du CMAP car rien de tout cela n'aurai été possible sans lui. Je sais qu'il appréciera la métaphore en disant qu'il fut pour moi un véritable "bâton de berger" durant ces trois années.

Je profite par ailleurs de ce préambule pour remercier ma famille, parents et amis. Je remercie tout particulièrement Sophie pour son soutient quotidien et ses encouragements continus. Elle eu le courage de démarrer la gestation de notre fils en même temps que je commençait celle de ce document. Mon fils arriva environ à la page 100 pour notre plus grand bonheur. Merci Martin pour tes réveilles nocturnes me forçant à poursuivre ma rédaction.

Finalement je vous remercie vous, cher lecteur de tenter l'aventure de lire ce document. J'imagine qu'il est peu probable que vous soyez mathématicien spécialiste en archéologie, architecture, modélisation graphique, acoustique et algorithmie, c'est pourquoi j'ai essayé d'être le plus clair possible dans chacune de mes parties pour permettre à chacun d'accéder sans encombre aux sujets où il sera néophytes. Dans tous les cas, je vous souhaite une bonne lecture.

