\chapter*{Introduction}
	\addcontentsline{toc}{chapter}{Introduction}
	
	
			
			La pluridisciplinarité consiste à aborder un objet d'étude selon les différents points de vue de regards spécialisés. Il s'agit de juxtaposer le travail de plusieurs disciplines autour d'un même objet d'étude. L'objectif de la pluridisciplinarité est ainsi d'utiliser la complémentarité intrinsèque des disciplines pour la résolution d'un problème. 
			2018 marque l'année de fusion entre l'\gls{upmc} et Paris-Sorbonne développant ainsi la transversalité et la collaboration entre de nombreux domaines scientifiques et littéraires. C'est pourquoi ce projet de thèse s'inscrit dans une démarche pluridisciplinaire en sollicitant les sciences mathématiques pour l'étude de problématiques archéologiques.
						
						
			En 2014, dans le cadre du projet NUMERO, les équipes d'archéologues de la Sorbonne et du \gls{cnrs} s'associent à l'\gls{iscd} de l'\gls{upmc} afin de virtualiser des fragments de décoration retrouvés dans les décombres du théâtre antique d'Orange. Cette collaboration a permis la numérisation de blocs issus de la frise du cortège dionysiaque ornant autrefois la façade du mur de scène et de pouvoir par la suite les rassembler virtuellement façon puzzle à l'aide d'un logiciel spécialement développé. "Il s'agit à terme, non seulement de restituer l'histoire du front de scène du théâtre et de comprendre les finalités de ses concepteurs, mais de fournir un corpus de référence pour l'ornementation architecturale en Narbonnaise" \citep{PouvoirDuTheatre}.
			
			Cette démarche a naturellement ouvert la voie à une étude plus large du théâtre d'Orange. L'objectif de cette thèse est ainsi de virtualiser le théâtre dans son ensemble afin de pouvoir en étudier l'architecture et les hypothèse de reconstitution de son état d'origine. Pour compléter cette démarche purement visuelle, l'équipe projet s'associe au \gls{cmap} pour étudier le comportement acoustique du théâtre. Comment virtualiser un monument d'une corpulence telle que celle du théâtre d'Orange (103 m de large pour 37 m de haut)? Comment remonter le temps pour restituer son architecture originel malgré les multiples transformations qu'il a subit durant des siècles? Comment en faire un outil numérique exploitable pour des études scientifiques divers? Quelles sont les solutions et les contraintes d'études acoustiques une tel lieu ? Quelles hypothèses archéologiques peut-on analyser par le biais d'une étude acoustique ? Comment diffuse-t-on des résultats pluridisciplinaire ? Voici la liste non-exhaustive des problématiques soulevées par un projet comme celui traité dans ce manuscrit de thèse. 