
%%%%%%%%%%%%%%%%%%%%%%%%%%%%%%%%%%%%%%%%
%           Liste des packages         %
%%%%%%%%%%%%%%%%%%%%%%%%%%%%%%%%%%%%%%%%


%% Faux texte, juste pour la démo
\usepackage{blindtext}

%interligne
\usepackage{setspace}
% police des chapitre
\usepackage{titlesec}
%\titleformat{\chapter}[display]{\bf\huge}{Chapitre \thechapter}{2pc}{}

%%%%%%%%%%%%%%%%%%%%%%%%%%%%%%%%%%%%%%%%%%%%%%%%%%%%%%%%%%%%%%%%%%%%%

%% Réglage des fontes et typo    
\usepackage[utf8]{inputenc}		% LaTeX, comprend les accents !
\usepackage[T1]{fontenc}

\usepackage[square,sort&compress,sectionbib]{natbib}		% Doit être chargé avant babel


\usepackage{chapterbib}
	\renewcommand{\bibsection}{\chapter*{Références}}		% Met les références biblio dans un \section (au lieu de \section*)
		
\usepackage[frenchb]{babel}
\usepackage{lmodern}
\usepackage{ae,aecompl}										% Utilisation des fontes vectorielles modernes
\usepackage[upright]{fourier}



%%%%%%%%%%%%%%%%%%%%%%%%%%%%%%%%%%%%%%%%%%%%%%%%%%%%%%%%%%%%%%%%%%%%%
% Allure générale du document
\usepackage{fancyhdr}			% Entête et pieds de page
	\pagestyle{fancy}			% Indique que le style de la page sera justement fancy
	\lfoot[\thepage]{} %gauche du pied de page
	\cfoot{} %milieu du pied de page
	\rfoot[]{\thepage} %droite du pied de page
	\fancyhead[RO, LE] {}	
\usepackage{enumerate}
\usepackage{enumitem}
\usepackage[section]{placeins}	% Place un FloatBarrier à chaque nouvelle section
\usepackage{epigraph}
\usepackage[font={small}]{caption}
\usepackage[francais,nohints]{minitoc}		% Mini table des matières, en français
	\setcounter{minitocdepth}{2}	% Mini-toc détaillées (sections/sous-sections)
%\usepackage[notbib]{tocbibind}		% Ajoute les Tables	des Matières/Figures/Tableaux à la table des matières

%%%%%%%%%%%%%%%%%%%%%%%%%%%%%%%%%%%%%%%%%%%%%%%%%%%%%%%%%%%%%%%%%%%%%

%% Maths                         
\usepackage{amsmath}			% Permet de taper des formules mathématiques
\usepackage{amssymb}			% Permet d'utiliser des symboles mathématiques
\usepackage{amsfonts}			% Permet d'utiliser des polices mathématiques
\usepackage{nicefrac}			% Fractions 'inline'


%%%%%%%%%%%%%%%%%%%%%%%%%%%%%%%%%%%%%%%%%%%%%%%%%%%%%%%%%%%%%%%%%%%%%


%% Tableaux
\usepackage{multirow}
\usepackage{booktabs}
\usepackage{colortbl}
\usepackage{tabularx}
\usepackage{multirow}
\usepackage{threeparttable}
\usepackage{etoolbox}
	\appto\TPTnoteSettings{\footnotesize}
\addto\captionsfrench{\def\tablename{{\textsc{Tableau}}}}	% Renome 'table' en 'tableau'

            
            

%%%%%%%%%%%%%%%%%%%%%%%%%%%%%%%%%%%%%%%%%%%%%%%%%%%%%%%%%%%%%%%%%%%%%
%% Graphiques                    
\usepackage{graphicx}			% Permet l'inclusion d'images
\usepackage{subcaption}
\usepackage{pdfpages}
\usepackage{rotating}
\usepackage{pgfplots}
	\usepgfplotslibrary{groupplots}
\usepackage{tikz}
	\usetikzlibrary{backgrounds,automata}
	\pgfplotsset{width=7cm,compat=1.3}
	\tikzset{every picture/.style={execute at begin picture={
   		\shorthandoff{:;!?};}
	}}
	\pgfplotsset{every linear axis/.append style={
		/pgf/number format/.cd,
		use comma,
		1000 sep={\,},
	}}
\usepackage{eso-pic}
\usepackage{import}
	
	



%%%%%%%%%%%%%%%%%%%%%%%%%%%%%%%%%%%%%%%%%%%%%%%%%%%%%%%%%%%%%%%%%%%%%
%% Mise en forme du texte        
\usepackage{xspace}
\usepackage[load-configurations = abbreviations]{siunitx}
	\DeclareSIUnit{\MPa}{\mega\pascal}
	\DeclareSIUnit{\micron}{\micro\meter}
	\DeclareSIUnit{\tr}{tr}
	\DeclareSIPostPower\totheM{m}
	\sisetup{
	locale = FR,
	  inter-unit-separator=$\cdot$,
	  range-phrase=~\`{a}~,     	% Utilise le tiret court pour dire "de... à"
	  range-units=single,  			% Cache l'unité sur la première borne
	  }
\usepackage{chemist}
\usepackage[version=3]{mhchem}
\usepackage{textcomp}
\usepackage{array}
\usepackage{hyphenat}



%%%%%%%%%%%%%%%%%%%%%%%%%%%%%%%%%%%%%%%%%%%%%%%%%%%%%%%%%%%%%%%%%%%%%
%% Navigation dans le document
\usepackage[pdftex,pdfborder={0 0 0},
			colorlinks=true,
			linkcolor=blue,
			citecolor=red,
			pagebackref=true,
			]{hyperref}	% Créera automatiquement les liens internes au PDF
						% Doit être chargé en dernier (Sauf exceptions)
			

%%%%%%%%%%%%%%%%%%%%%%%%%%%%%%%%%%%%%%%%%%%%%%%%%%%%%%%%%%%%%%%%%%%%%
%% Packages qui doivent être chargés APRES hyperref	       
      
\usepackage[top=2.5cm, bottom=2cm, left=3cm, right=2.5cm,
			headheight=15pt]{geometry}
\usepackage[acronym,xindy,toc,numberedsection]{glossaries}
	\newglossary[nlg]{notation}{not}{ntn}{Notation} 	% Création d'un type de glossaire 'notation'
	\makeglossaries
	\loadglsentries{Glossaire}							% Utilisation d'un fichier externe pour la définition des entrées (Glossaire.tex)		
	
	
			