\part{Calculs acoustiques}
	\chapter*{Introduction}
	\addcontentsline{toc}{chapter}{Introduction}
	
	\chapter{Acoustique de salle}
		\citationChap{
			Et c'est là que jadis, à quinze ans révolus\\
			A l'âge où s'amuser tout seul ne suffit plus\\
			Je connus la prime amourette\\
			Auprès d'une sirène, une femme-poisson\\
			Je reçus de l'amour la première leçon\\
			Avalai la première arête
		}{Georges Brassens}
		\minitoc
		\newpage
		
		\section{Introduction à l'acoustique de salle}
		Livre de Jouhaneau
		\section{Méthodes de calcul acoustique}
			\subsection{FEM/BEM}
			\subsection{RayTracing}
			\subsection{Sources-image}
			\subsection{Statistique}
		\section{Méthode couplée}
		
	\chapter{Logiciel développé}
		\citationChap{
		Sucette à la viande
		}{MC Grobi}
		\minitoc
		\newpage
		\section{Introduction}
		Rappel des problématiques, conditions choisies (source omni, parois non diffusante)
		\section{Lecture de maillage}
		\section{Calcul de rayon}
		\section{Calcul de sources-images}
		\section{Génération de réponse impulsionnel}
		\section{Méthode d'octree}
		\section{Vue d'ensemble}
		Add-on blender, traitement du signal, etc

	\chapter{Validation}
		\citationChap{
		Areuh
		}{Ilde Flucki}
		\minitoc
		\newpage
		\section{Introduction}
		Rappel cahier des charges
		\section{Comparaison aux cas test}
			\subsection{Salle rectangulaire}
			\subsection{Salle sphèrique}
		\section{Analyse de complexité}
		
	\chapter*{Conclusion}
	\addcontentsline{toc}{chapter}{Conclusion}
		\newpage
		
% Biblio
 \bibliographystyle{francaissc}
 \bibliography{Part2/Biblio}
\addcontentsline{toc}{chapter}{Références}
